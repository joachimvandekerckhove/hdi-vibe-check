
\begin{equation*}
    P(X - 1) =
    \begin{cases}
    \gamma + (1-\gamma)\frac{1}{1+e'[-\eta]} & \text{Item Response Theory} \\ %That is an ugly negation slash but I can't find a better looking one atm
    q + (1-q)b & \text{Multinomial Processing Tree}.
    \end{cases}
\end{equation*}

q is the probability that you know the item
1-q you don't know the item but guess (b)
where $\gamma$ and $q$ are unknown but we want to estimate it.

\begin{equation*}    
    \gamma &= b
\end{equation*}

\begin{equation*}
    q &= \frac{1}{1+e'{-\eta}}
\end{equation*}


\begin{equation*}
J = \begin{bmatrix}
1 & 0 \\
0 & \frac{1}{q(1-q)} 
\end{bmatrix}
\end{equation*}

eta is the difference entween item difficulty and participants ability

Items own dificulty
Persons own ability

eta would have subscripts ip, but there are not ixp of them, there are i+p of them

The dominance parameter is always split between person and item

Three items, two people










+++++++++++++++++

Time was coded as a continuous variable representing the number of years since baseline,
so that the baseline, 6-, 12-, 24- and 36-month measurements were coded as 0, 0.5, 1, 2 and 3 respectively. The effects of the interventions on change in outcomes over
time were assessed through Group × Time interaction
terms. These models take into account individual differences 
at baseline, using baseline measurements as the
reference point to estimate participant-specific starting
points and change over time from these baseline levels.
Growth functions (i.e. the nature of change in outcomes
over time) were assessed by testing alternative functions
in unconditional models which did not include group pre-
dictors. Linear and quadratic time coefficients were suc-
cessively added to the models, and likelihood-ratio tests
were used to determine whether the term improved model
fit for that outcome.
Random intercepts and slopes were estimated at the
individual level (nested within schools), and random in-
tercepts at the school level. The best-fitting random ef-
fects structure for each outcome was tested using
likelihood ratio tests, with Akaike information criterion
(AIC) statistics used to confirm these comparisons since
likelihood ratio tests regarding these parameters are con-
servative under some conditions [41]. The binary out-
comes of cannabis use and cannabis harms were
modelled using mixed-effects logistic regression via Sta-
ta’s melogit command, while the continuous outcome of
cannabis knowledge was modelled using the Stata mixed
command. In order to interpret the estimated
changes, the margins and lincom commands were used
to obtain predicted group means and differences at each
measurement occasion.


 For cannabis knowledge, the HDI did not overlap the region of practical equivalence for CAP or Climate at the 6, 12 and 24-month follow-ups, showing evidence of an intervention effect, but overlapped at 36 months, showing insufficient evidence to conclude in favour a meaningful intervention effect or no difference.

The authors 

The \hdr{} was also implemented to approach the effect size $d$, with a test value $d=1$ and a corresponding ROPE of $[-0.1,0.1]$
\

 For canna-
bis knowledge, there was evidence of an intervention effect
at the 6, 12 and 24-month follow-ups, but at 36 months
there was insufficient evidence to conclude in favour of
either a meaningful intervention effect or no differenc
