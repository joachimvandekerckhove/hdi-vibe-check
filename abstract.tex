The Bayesian HDI+ROPE decision rule is an increasingly common approach to testing null parameter values. 
The decision procedure involves a comparison between a posterior highest density interval (HDI) and a pre-specified region of practical equivalence (ROPE). One then accepts or rejects the null parameter value depending on the overlap (or lack thereof) between these intervals. 
Here we demonstrate, both theoretically and through examples, that this procedure is logically incoherent.
Because the HDI is not transformation invariant, the ultimate inferential decision depends on statistically arbitrary and scientifically irrelevant properties of the statistical model.
The incoherence arises from a common confusion between probability density and probability proper.  The HDI+ROPE procedure relies on characterizing posterior densities as opposed to being based directly on probability. 
We conclude with recommendations for alternative Bayesian testing procedures that do not exhibit this pathology and provide a ``quick fix'' in the form of quantile intervals.